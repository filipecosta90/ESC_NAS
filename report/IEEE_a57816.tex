
% !TEX encoding = UTF-8 Unicode


%% bare_conf_compsoc.tex
%% V1.4b
%% 2015/08/26
%% by Michael Shell
%% See:
%% http://www.michaelshell.org/
%% for current contact information.
%%
%% This is a skeleton file demonstrating the use of IEEEtran.cls
%% (requires IEEEtran.cls version 1.8b or later) with an IEEE Computer
%% Society conference paper.
%%
%% Support sites:
%% http://www.michaelshell.org/tex/ieeetran/
%% http://www.ctan.org/pkg/ieeetran
%% and
%% http://www.ieee.org/

%%*************************************************************************
%% Legal Notice:
%% This code is offered as-is without any warranty either expressed or
%% implied; without even the implied warranty of MERCHANTABILITY or
%% FITNESS FOR A PARTICULAR PURPOSE! 
%% User assumes all risk.
%% In no event shall the IEEE or any contributor to this code be liable for
%% any damages or losses, including, but not limited to, incidental,
%% consequential, or any other damages, resulting from the use or misuse
%% of any information contained here.
%%
%% All comments are the opinions of their respective authors and are not
%% necessarily endorsed by the IEEE.
%%
%% This work is distributed under the LaTeX Project Public License (LPPL)
%% ( http://www.latex-project.org/ ) version 1.3, and may be freely used,
%% distributed and modified. A copy of the LPPL, version 1.3, is included
%% in the base LaTeX documentation of all distributions of LaTeX released
%% 2003/12/01 or later.
%% Retain all contribution notices and credits.
%% ** Modified files should be clearly indicated as such, including  **
%% ** renaming them and changing author support contact information. **
%%*************************************************************************


% *** Authors should verify (and, if needed, correct) their LaTeX system  ***
% *** with the testflow diagnostic prior to trusting their LaTeX platform ***
% *** with production work. The IEEE's font choices and paper sizes can   ***
% *** trigger bugs that do not appear when using other class files.       ***                          ***
% The testflow support page is at:
% http://www.michaelshell.org/tex/testflow/



\documentclass[conference,compsoc]{IEEEtran}
% Some/most Computer Society conferences require the compsoc mode option,
% but others may want the standard conference format.
%
% If IEEEtran.cls has not been installed into the LaTeX system files,
% manually specify the path to it like:
% \documentclass[conference,compsoc]{../sty/IEEEtran}





% Some very useful LaTeX packages include:
% (uncomment the ones you want to load)


% *** MISC UTILITY PACKAGES ***
%
%\usepackage{ifpdf}
% Heiko Oberdiek's ifpdf.sty is very useful if you need conditional
% compilation based on whether the output is pdf or dvi.
% usage:
% \ifpdf
%   % pdf code
% \else
%   % dvi code
% \fi
% The latest version of ifpdf.sty can be obtained from:
% http://www.ctan.org/pkg/ifpdf
% Also, note that IEEEtran.cls V1.7 and later provides a builtin
% \ifCLASSINFOpdf conditional that works the same way.
% When switching from latex to pdflatex and vice-versa, the compiler may
% have to be run twice to clear warning/error messages.






% *** CITATION PACKAGES ***
%
\ifCLASSOPTIONcompsoc
  % IEEE Computer Society needs nocompress option
  % requires cite.sty v4.0 or later (November 2003)
  \usepackage[nocompress]{cite}
\else
  % normal IEEE
  \usepackage{cite}
\fi
% cite.sty was written by Donald Arseneau
% V1.6 and later of IEEEtran pre-defines the format of the cite.sty package
% \cite{} output to follow that of the IEEE. Loading the cite package will
% result in citation numbers being automatically sorted and properly
% "compressed/ranged". e.g., [1], [9], [2], [7], [5], [6] without using
% cite.sty will become [1], [2], [5]--[7], [9] using cite.sty. cite.sty's
% \cite will automatically add leading space, if needed. Use cite.sty's
% noadjust option (cite.sty V3.8 and later) if you want to turn this off
% such as if a citation ever needs to be enclosed in parenthesis.
% cite.sty is already installed on most LaTeX systems. Be sure and use
% version 5.0 (2009-03-20) and later if using hyperref.sty.
% The latest version can be obtained at:
% http://www.ctan.org/pkg/cite
% The documentation is contained in the cite.sty file itself.
%
% Note that some packages require special options to format as the Computer
% Society requires. In particular, Computer Society  papers do not use
% compressed citation ranges as is done in typical IEEE papers
% (e.g., [1]-[4]). Instead, they list every citation separately in order
% (e.g., [1], [2], [3], [4]). To get the latter we need to load the cite
% package with the nocompress option which is supported by cite.sty v4.0
% and later.





% *** GRAPHICS RELATED PACKAGES ***
%
\ifCLASSINFOpdf
  % \usepackage[pdftex]{graphicx}
  % declare the path(s) where your graphic files are
  % \graphicspath{{../pdf/}{../jpeg/}}
  % and their extensions so you won't have to specify these with
  % every instance of \includegraphics
  % \DeclareGraphicsExtensions{.pdf,.jpeg,.png}
\else
  % or other class option (dvipsone, dvipdf, if not using dvips). graphicx
  % will default to the driver specified in the system graphics.cfg if no
  % driver is specified.
  % \usepackage[dvips]{graphicx}
  % declare the path(s) where your graphic files are
  % \graphicspath{{../eps/}}
  % and their extensions so you won't have to specify these with
  % every instance of \includegraphics
  % \DeclareGraphicsExtensions{.eps}
\fi
% graphicx was written by David Carlisle and Sebastian Rahtz. It is
% required if you want graphics, photos, etc. graphicx.sty is already
% installed on most LaTeX systems. The latest version and documentation
% can be obtained at: 
% http://www.ctan.org/pkg/graphicx
% Another good source of documentation is "Using Imported Graphics in
% LaTeX2e" by Keith Reckdahl which can be found at:
% http://www.ctan.org/pkg/epslatex
%
% latex, and pdflatex in dvi mode, support graphics in encapsulated
% postscript (.eps) format. pdflatex in pdf mode supports graphics
% in .pdf, .jpeg, .png and .mps (metapost) formats. Users should ensure
% that all non-photo figures use a vector format (.eps, .pdf, .mps) and
% not a bitmapped formats (.jpeg, .png). The IEEE frowns on bitmapped formats
% which can result in "jaggedy"/blurry rendering of lines and letters as
% well as large increases in file sizes.
%
% You can find documentation about the pdfTeX application at:
% http://www.tug.org/applications/pdftex





% *** MATH PACKAGES ***
%
%\usepackage{amsmath}
% A popular package from the American Mathematical Society that provides
% many useful and powerful commands for dealing with mathematics.
%
% Note that the amsmath package sets \interdisplaylinepenalty to 10000
% thus preventing page breaks from occurring within multiline equations. Use:
%\interdisplaylinepenalty=2500
% after loading amsmath to restore such page breaks as IEEEtran.cls normally
% does. amsmath.sty is already installed on most LaTeX systems. The latest
% version and documentation can be obtained at:
% http://www.ctan.org/pkg/amsmath





% *** SPECIALIZED LIST PACKAGES ***
%
%\usepackage{algorithmic}
% algorithmic.sty was written by Peter Williams and Rogerio Brito.
% This package provides an algorithmic environment fo describing algorithms.
% You can use the algorithmic environment in-text or within a figure
% environment to provide for a floating algorithm. Do NOT use the algorithm
% floating environment provided by algorithm.sty (by the same authors) or
% algorithm2e.sty (by Christophe Fiorio) as the IEEE does not use dedicated
% algorithm float types and packages that provide these will not provide
% correct IEEE style captions. The latest version and documentation of
% algorithmic.sty can be obtained at:
% http://www.ctan.org/pkg/algorithms
% Also of interest may be the (relatively newer and more customizable)
% algorithmicx.sty package by Szasz Janos:
% http://www.ctan.org/pkg/algorithmicx




% *** ALIGNMENT PACKAGES ***
%
%\usepackage{array}
% Frank Mittelbach's and David Carlisle's array.sty patches and improves
% the standard LaTeX2e array and tabular environments to provide better
% appearance and additional user controls. As the default LaTeX2e table
% generation code is lacking to the point of almost being broken with
% respect to the quality of the end results, all users are strongly
% advised to use an enhanced (at the very least that provided by array.sty)
% set of table tools. array.sty is already installed on most systems. The
% latest version and documentation can be obtained at:
% http://www.ctan.org/pkg/array


% IEEEtran contains the IEEEeqnarray family of commands that can be used to
% generate multiline equations as well as matrices, tables, etc., of high
% quality.




% *** SUBFIGURE PACKAGES ***
%\ifCLASSOPTIONcompsoc
%  \usepackage[caption=false,font=footnotesize,labelfont=sf,textfont=sf]{subfig}
%\else
%  \usepackage[caption=false,font=footnotesize]{subfig}
%\fi
% subfig.sty, written by Steven Douglas Cochran, is the modern replacement
% for subfigure.sty, the latter of which is no longer maintained and is
% incompatible with some LaTeX packages including fixltx2e. However,
% subfig.sty requires and automatically loads Axel Sommerfeldt's caption.sty
% which will override IEEEtran.cls' handling of captions and this will result
% in non-IEEE style figure/table captions. To prevent this problem, be sure
% and invoke subfig.sty's "caption=false" package option (available since
% subfig.sty version 1.3, 2005/06/28) as this is will preserve IEEEtran.cls
% handling of captions.
% Note that the Computer Society format requires a sans serif font rather
% than the serif font used in traditional IEEE formatting and thus the need
% to invoke different subfig.sty package options depending on whether
% compsoc mode has been enabled.
%
% The latest version and documentation of subfig.sty can be obtained at:
% http://www.ctan.org/pkg/subfig




% *** FLOAT PACKAGES ***
%
%\usepackage{fixltx2e}
% fixltx2e, the successor to the earlier fix2col.sty, was written by
% Frank Mittelbach and David Carlisle. This package corrects a few problems
% in the LaTeX2e kernel, the most notable of which is that in current
% LaTeX2e releases, the ordering of single and double column floats is not
% guaranteed to be preserved. Thus, an unpatched LaTeX2e can allow a
% single column figure to be placed prior to an earlier double column
% figure.
% Be aware that LaTeX2e kernels dated 2015 and later have fixltx2e.sty's
% corrections already built into the system in which case a warning will
% be issued if an attempt is made to load fixltx2e.sty as it is no longer
% needed.
% The latest version and documentation can be found at:
% http://www.ctan.org/pkg/fixltx2e


%\usepackage{stfloats}
% stfloats.sty was written by Sigitas Tolusis. This package gives LaTeX2e
% the ability to do double column floats at the bottom of the page as well
% as the top. (e.g., "\begin{figure*}[!b]" is not normally possible in
% LaTeX2e). It also provides a command:
%\fnbelowfloat
% to enable the placement of footnotes below bottom floats (the standard
% LaTeX2e kernel puts them above bottom floats). This is an invasive package
% which rewrites many portions of the LaTeX2e float routines. It may not work
% with other packages that modify the LaTeX2e float routines. The latest
% version and documentation can be obtained at:
% http://www.ctan.org/pkg/stfloats
% Do not use the stfloats baselinefloat ability as the IEEE does not allow
% \baselineskip to stretch. Authors submitting work to the IEEE should note
% that the IEEE rarely uses double column equations and that authors should try
% to avoid such use. Do not be tempted to use the cuted.sty or midfloat.sty
% packages (also by Sigitas Tolusis) as the IEEE does not format its papers in
% such ways.
% Do not attempt to use stfloats with fixltx2e as they are incompatible.
% Instead, use Morten Hogholm'a dblfloatfix which combines the features
% of both fixltx2e and stfloats:
%
% \usepackage{dblfloatfix}
% The latest version can be found at:
% http://www.ctan.org/pkg/dblfloatfix




% *** PDF, URL AND HYPERLINK PACKAGES ***
%
%\usepackage{url}
% url.sty was written by Donald Arseneau. It provides better support for
% handling and breaking URLs. url.sty is already installed on most LaTeX
% systems. The latest version and documentation can be obtained at:
% http://www.ctan.org/pkg/url
% Basically, \url{my_url_here}.




% *** Do not adjust lengths that control margins, column widths, etc. ***
% *** Do not use packages that alter fonts (such as pslatex).         ***
% There should be no need to do such things with IEEEtran.cls V1.6 and later.
% (Unless specifically asked to do so by the journal or conference you plan
% to submit to, of course. )


% correct bad hyphenation here
\hyphenation{op-tical net-works semi-conduc-tor}

\usepackage[utf8x]{inputenc} 


\usepackage{array}
\newcolumntype{L}[1]{>{\raggedright\let\newline\\\arraybackslash\hspace{0pt}}m{#1}}
\newcolumntype{C}[1]{>{\centering\let\newline\\\arraybackslash\hspace{0pt}}m{#1}}
\newcolumntype{R}[1]{>{\raggedleft\let\newline\\\arraybackslash\hspace{0pt}}m{#1}}

\usepackage{float}


\begin{document}
%
% paper title
% Titles are generally capitalized except for words such as a, an, and, as,
% at, but, by, for, in, nor, of, on, or, the, to and up, which are usually
% not capitalized unless they are the first or last word of the title.
% Linebreaks \\ can be used within to get better formatting as desired.
% Do not put math or special symbols in the title.
\title{Introdução ao NAS Parallel Benchmarks\\ Performance Relativa de Kernels Sequenciais, em ambiente de Memória Partilhada e ambiente de Memória Distribuída}


% author names and affiliations
% use a multiple column layout for up to three different
% affiliations
\author{\IEEEauthorblockN{Filipe Oliveira}
\IEEEauthorblockA{Departamento de Informática\\
Universidade do Minho\\
Email: a57816@alunos.uminho.pt}
}

% conference papers do not typically use \thanks and this command
% is locked out in conference mode. If really needed, such as for
% the acknowledgment of grants, issue a \IEEEoverridecommandlockouts
% after \documentclass

% for over three affiliations, or if they all won't fit within the width
% of the page (and note that there is less available width in this regard for
% compsoc conferences compared to traditional conferences), use this
% alternative format:
% 
%\author{\IEEEauthorblockN{Michael Shell\IEEEauthorrefmark{1},
%Homer Simpson\IEEEauthorrefmark{2},
%James Kirk\IEEEauthorrefmark{3}, 
%Montgomery Scott\IEEEauthorrefmark{3} and
%Eldon Tyrell\IEEEauthorrefmark{4}}
%\IEEEauthorblockA{\IEEEauthorrefmark{1}School of Electrical and Computer Engineering\\
%Georgia Institute of Technology,
%Atlanta, Georgia 30332--0250\\ Email: see http://www.michaelshell.org/contact.html}
%\IEEEauthorblockA{\IEEEauthorrefmark{2}Twentieth Century Fox, Springfield, USA\\
%Email: homer@thesimpsons.com}
%\IEEEauthorblockA{\IEEEauthorrefmark{3}Starfleet Academy, San Francisco, California 96678-2391\\
%Telephone: (800) 555--1212, Fax: (888) 555--1212}
%\IEEEauthorblockA{\IEEEauthorrefmark{4}Tyrell Inc., 123 Replicant Street, Los Angeles, California 90210--4321}}




% use for special paper notices
%\IEEEspecialpapernotice{(Invited Paper)}




% make the title area
\maketitle

% As a general rule, do not put math, special symbols or citations
% in the abstract
\begin{abstract}

Neste estudo, analisamos a performance de kernels 
\end{abstract}

% no keywords




% For peer review papers, you can put extra information on the cover
% page as needed:
% \ifCLASSOPTIONpeerreview
% \begin{center} \bfseries EDICS Category: 3-BBND \end{center}
% \fi
%
% For peerreview papers, this IEEEtran command inserts a page break and
% creates the second title. It will be ignored for other modes.
\IEEEpeerreviewmaketitle



\section{Introduction}
% no \IEEEPARstart
This demo file is intended to serve as a ``starter file''
for IEEE Computer Society conference papers produced under \LaTeX\ using
IEEEtran.cls version 1.8b and later.
% You must have at least 2 lines in the paragraph with the drop letter
% (should never be an issue)
I wish you the best of success.

\hfill Filipe Oliveira
 
\hfill 1 Março, 2016

\section{Contextualização das Benchmarks}
%NAS Parallel Benchmarks (NPB’s) [1] were derived from CFD codes. They were designed to
%compare the performance of parallel computers and are widely recognized as a standard indicator
%of computer performance. 
As "NAS Parallel Benchmarks"\cite{nas} englobam 5 kernels (EP, MG, CG, FT, IS) e 3 aplicações  que simulam dinâmica de fluídos (LU,SP,BT). No nosso caso de estudo, temos por interesse os 5 kernels, dado que centraremos o nosso estudo da performance relativa via alterações no paradigma de memória e forma de comunicação entre nodos, assim como a própria ferramenta de compilação e respectivas flags. Assim sendo, temos então 5 opções a analisar: EP, MG, CG, FT, IS. Resta-nos portanto analisar primeiramente quais as principais propriedades de cada um antes de qualquer avanço no trabalho. \par 
\begin{itemize}

\item \textbf{EP},  tal como o próprio nome indica (Embarrassingly Parallel), que por calcular números aleatórios é um kernel implicitamente embaracosamente paralelo. Tem como propósito estabelecer o Peak Performance em "FP Operations" de um sistema de computação em teste. É então espectável obtermos os melhores resultados de performance neste kernel e, por esse mesmo motivo, este será um dos kernels com grande relevância para o nosso caso de estudo.

\item \textbf{MG}, cujo kernel  implementa um método numérico multigrid simplificado, numa sequência de malhas de diferentes propriedades, implicando portanto uma elevada comunicação para a resolução do algoritmo. Tanto para as versões em memória distribuída com para a versão de memória partilhada será interessante analisar o comportamento do kernel nos ambientes de teste. Será portanto também incluído no caso de estudo.

\item \textbf{CG},que recorre ao método do Conjugado do Gradiente por forma a calcular uma aproximação ao menor dos valores próprios de uma matriz esparsa de grandes dimensões. Dada o tipo de dados, este kernel testa computação e comunicação não estruturada, sendo portanto expectável uma fraca performance deste kernel quando em comparação com o \textbf{EP}. Será portanto também incluído no caso de estudo.

\item \textbf{FT}, que calcula a Transformada de Fourier em 3 dimensões (3 transformadas de Fourier de uma dimensão), sendo o resultado a solução de uma equação diferencial parcial. Dado que a principal propriedade a ser estudada é comunicação, este kernel será portanto excluído do caso de estudo em detrimento do \textbf{MG}.


\item \textbf{IS}, que realiza operações de sorting em inteiros. Este kernel testa tanto a capacidade de computação de um sistema em termos de operações sobre inteiros, assim como a performance de comunicação do mesmo dada a irregularidade dos acessos à memória e, quando aplicável, comunicação entre processos. Será também incluído no caso de estudo.

 \end{itemize}



\section{Caracterização do Hardware do ambiente de testes}
Escolhidas as benchmarks, resta-me especificar os ambientes de teste nos quais pretendo realizar as benchmarks. Através da análise do hardware disponível no Search6 \footnote{Services and Advanced Research Computing with HTC/HPC clusters}, a nossa plataforma de teste, deparamo-nos com duas 
realidades distintas presentes no mesmo cluster. Se por um lado temos uma grande porção dos nodos de computação com configurações de hardware relativamente homogéneas (28 dos 54 nodos disponíveis apresentam todos as mesmas características de comunicação, armazenamento local, memória RAM disponível e mesma família de processadores - Ivy Bridge), por outro lado temos os restantes 26 nodos com características relativamente distintas entre nodos (diferentes famílias de processadores, diferentes características de comunicação entre nodos disponíveis, memória RAM disponível em diferente número). \par 
Considerando a abrangência do número de nodos o ponto essencial para a escolha do tipo de nodos a teste, decidi incluir no mesmo os nodos do tipo 662, 652, 641, e 431(apenas os nodos com 48GB de memória RAM disponíveis), que passarei de seguida a caracterizar. \par 
Denote que ao realizarmos os testes de performance nos nodos acima enumerados estamos a abrangir 33 dos 54 nodos disponíveis, preservando características entre eles fundamentais para a possibilidade de comparação como por exemplo o suporte da rede Myrinet 10Gbps, e englobando como requerido mais do que uma classe de arquitectura existe no Search6.\par 


\begin{table}[H]
\caption{Características de Hardware do nodo 662}
     \label{table:characterization}
\centering
  \begin{tabular}{ | l | r | }
  
    \hline
    Sistema & compute-662 \\ \hline \hline
        \# CPUs & 2  \\ \hline
    CPU & Intel\textsuperscript{\textregistered} Xeon\textsuperscript{\textregistered} E5-2695 v2  \\ \hline 
    Arquitectura de Processador & Ivy Bridge  \\ \hline 
    \# Cores por CPU & 12   \\ \hline 
    \# Threads por CPU & 24  \\ \hline 
     Freq. Clock & 2.4 GHz  \\ \hline
    Cache L1  & 384KB  (32KB por Core)  \\ \hline 
    Cache L2  & 3072KB (256KB por Core)  \\ \hline 
    Cache L3  & 30720KB (partilhada) \\ \hline 
    Ext. Inst. Set  & SSE4.2, AVX  \\ \hline 
        \#Memory Channels & 4 \\ \hline
        Memória Ram Disponível & 64GB \\ \hline
     Peak Memory BW Fab. CPU  & 59.7 GB/s \\ \hline
%    Peak Memory BW Medida\cite{stream}  & 25.1GB/s \\ \hline
            Rede Suportada  & Gigabit Ethernet, Myrinet 10Gbps \\ \hline
  \end{tabular}
\end{table}

\begin{table}[H]
\caption{Características de Hardware do nodo 652}
     \label{table:characterization}
\centering
  \begin{tabular}{ | l | r | }
  
    \hline
    Sistema & compute-652 \\ \hline \hline
        \# CPUs & 2  \\ \hline
    CPU & Intel\textsuperscript{\textregistered} Xeon\textsuperscript{\textregistered} E5-2670 v2 \\ \hline 
    Arquitectura de Processador & Ivy Bridge  \\ \hline 
    \# Cores por CPU & 10   \\ \hline 
    \# Threads por CPU & 20  \\ \hline 
     Freq. Clock & 2.5 GHz  \\ \hline
    Cache L1  & 320KB  (32KB por Core)  \\ \hline 
    Cache L2  & 2560KB (256KB por Core)  \\ \hline 
    Cache L3  & 25600KB (partilhada) \\ \hline 
    Ext. Inst. Set  & SSE4.2, AVX  \\ \hline 
         \#Memory Channels & 4 \\ \hline
        Memória Ram Disponível & 64GB \\ \hline
     Peak Memory BW Fab. CPU  & 59.7 GB/s \\ \hline
%    Peak Memory BW Medida\cite{stream}  & 25.1GB/s \\ \hline
            Rede Suportada  & Gigabit Ethernet, Myrinet 10Gbps \\ \hline
  \end{tabular}
\end{table}

\begin{table}[H]
\caption{Características de Hardware do nodo 641}
     \label{table:characterization}
\centering
  \begin{tabular}{ | l | r | }
  
    \hline
    Sistema & compute-641 \\ \hline \hline
        \# CPUs & 2  \\ \hline
    CPU & Intel\textsuperscript{\textregistered} Xeon\textsuperscript{\textregistered}  E5-2650 v2 \\ \hline 
    Arquitectura de Processador & Ivy Bridge  \\ \hline 
    \# Cores por CPU & 8   \\ \hline 
    \# Threads por CPU & 16  \\ \hline 
     Freq. Clock & 2.6 GHz  \\ \hline
    Cache L1  & 256KB  (32KB por Core)  \\ \hline 
    Cache L2  & 2048KB (256KB por Core)  \\ \hline 
    Cache L3  & 20480KB (partilhada) \\ \hline 
  Ext. Inst. Set  & SSE4.2, AVX  \\ \hline 
         \#Memory Channels & 4 \\ \hline
        Memória Ram Disponível & 64GB \\ \hline
     Peak Memory BW Fab. CPU  & 59.7 GB/s \\ \hline
%    Peak Memory BW Medida\cite{stream}  & 25.1GB/s \\ \hline
            Rede Suportada  & Gigabit Ethernet, Myrinet 10Gbps \\ \hline
  \end{tabular}
\end{table}

\begin{table}[h!]
\caption{Características de Hardware do nodo 431}
     \label{table:characterization}
\centering
  \begin{tabular}{ | l | r | }
  
    \hline
    Sistema & compute-431 \\ \hline \hline
        \# CPUs & 2  \\ \hline
    CPU & Intel\textsuperscript{\textregistered} Xeon\textsuperscript{\textregistered} X5650 \\ \hline 
    Arquitectura de Processador & Nehalem  \\ \hline 
    \# Cores por CPU & 6   \\ \hline 
    \# Threads por CPU & 12  \\ \hline 
     Freq. Clock & 2.66 GHz  \\ \hline
    Cache L1  & 192KB  (32KB por Core)  \\ \hline 
    Cache L2  & 1536KB (256KB por Core)  \\ \hline 
    Cache L3  & 12288KB (partilhada) \\ \hline 
    Ext. Inst. Set  & SSE4.2   \\ \hline 
        \#Memory Channels & 3 \\ \hline
        Memória Ram Disponível & 48GB \\ \hline
     Peak Memory BW Fab. CPU  & 32 GB/s \\ \hline
%    Peak Memory BW Medida\cite{stream}  & 25.1GB/s \\ \hline
            Rede Suportada  & Gigabit Ethernet, Myrinet 10Gbps \\ \hline
  \end{tabular}
\end{table}

Da caracterização de hardware acima realizada podemos retirar já as diferentes possibilidades relativamente ao número de threads a testar em ambiente de memória partilhada, assim como propriedades adequadas aos testes em ambiente de memória distribuída. Para além do anteriormente enumerado  obtive dados de extrema importância relativa à próxima decisão do nosso caso de estudo -- a escolha das classes de dados a incluir no nosso caso de estudo. Será com base nas propriedades relativas à Memória RAM disponível, tamanho e forma de distribuição dos vários níveis de cache, assim como a Peak Memory Bandwith teórica de Memória do CPU, que centrarei a minha decisão sobre quais as classes de dados relevantes.

\subsection{Inclusão de diferentes dimensões (classes) de dados}

Será importante para a relevância dos testes de perfomance incluir datasets de diferentes dimensões. Ora, analisando as propriedades dos processadores presentes nos nossos nodos de computação devo portante seleccionar um dataset capaz de ser compreendido na Cache L1 (menor ou igual a 32KB), um dataset capaz de ser compreendido na Cache L2 (menor ou igual a 256KB), um dataset capaz de ser compreendido na Cache L3 (menor que 12MB) e um dataset capaz de ser compreendido na Memória Ram disponível nos nodos de computação (menor que 48GB). \par 
Analisadas as dimensões dos problemas para as diferentes Classes de dados e Benchmarks a incluir no nosso caso de estudo, verificamos que em nenhum dos casos os datasets são possíveis de ser compreendidos totalmente na cache L1. Assim, e focando o nosso processo de seleção com base nas benchmarks IS e MG podemos concluir que as classes de dados que melhor se associam ao nosso problema são portanto a classe S (totalmente contida na cache L2 ou L3 ), e as classes A, B, e C por serem consideradas classes standard das benchmarks e de fácil comparação de resultados com a comunidade do HPC. \par 
Denote ainda que para o caso da benchmark CG a classe de dados C apresenta um tamanho de dados superior ao possível de estar contido na memória principal obrigando teoricamente a um decréscimo de perfomance derivada do aumento de IO.\par 
No anexo X presente na página X poderá encontrar informação mais detalhada relativa à forma de cálculo do tamanho dos datasets das diferentes classes de dados. 

\begin{table}[h!]
\caption{Dimensão do dataset para as diferentes Classes e Benchmarks}
     \label{table:dimensaoproblema}
\centering
  \begin{tabular}{ | l | r |  r | r | r | r |  }
  
    \hline
    Bench. & data type & S & A & B & C \\ \hline 
    
     EP &  double & 128	MB & 2	GB	 & 8	GB	& 32	GB \\ \hline 
     
  MG &  double &   256	KB & 128	MB	& 128	MB	& 1024	MB \\ \hline 
    CG & double & 15MB & 1,46	GB& 41,91	GB	& 167,64	GB \\ \hline 
    IS & integer & 256	KB & 32	MB	& 128	MB	& 512	MB \\ \hline 

  \end{tabular}
\end{table}

\section{Caracterização do Software do ambiente de testes}




%\subsection{Inclusão de diferentes módulos de software e ferramentas}
O cluster Search6 é baseado no Rocks 6.1 cluster management distribution. 
Dado que  um dos principais propósitos deste caso de estudo acenta na investigação da influência de diferentes ferramentas de compilação e diferentes configurações de ferramentas de comunicação, assim como dos diferentes paradigmas de memória, na perfomance global dos kernels, o próximo passo natural passa por identificar os módulos disponíveis no nosso ambiente de clustering que cumprem o objectivo anteriormente enumerado. \par 
Relativamente aos compiladores de interesse para o caso de estudo temos portanto o \textbf{GCC compiler suite}	e o \textbf{ Intel compilers suite}. O último apresenta apenas a possibilidade de seleção da versão "icc version 13.0.1 (gcc version 4.4.6 compatibility)". Ora, pela própria informação presente na versão do compilador da Intel deveremos incluir no nosso leque de testes a versão do compilador da gnu 4.4.6. Irei também incluir a versão 4.9.0 do compilador da Gnu por ser considera a versão default deste mesmo compilador no nosso ambiente de clustering.\par 
Relativamente às MPI stacks, nomeadamente à versão do OpenMPI a ser incluída, será em ambos os casos (Intel e GNU) a versão mais recente disponível para ambos, nomeadamente a versão 1.8.2 dos módulos com via de comunicação Gigabit Ethernet e Myrinet 10Gbps.\par 
Relativamente às flags de compilação serão testas as respectivas optimizações disponibilizadas pelas flags de compilação -O2 e -O3, e sem qualquer tipo de otimização por flags de compilação.\par 
Teremos portanto as seguintes combinações de software distintas no nosso caso de estudo:

\begin{itemize}
\item Compiladores distintos:
\begin{itemize}
\item Kernels compilados com compilar da GNU gcc versão 4.4.6
\item Kernels compilados com compilar da GNU gcc versão 4.9.0
\item Kernels compilados com compilar da Intel icc versão 13.0.1
\end{itemize}
\item Versão do OpenMPI:
\begin{itemize}
\item versão 1.8.2 via comunicação Gigabit Ethernet para compilador da GNU
\item versão 1.8.2 via comunicação Gigabit Ethernet para compilador da Intel
\item versão 1.8.2 via comunicação Myrinet 10Gbps para compilador da GNU
\item versão 1.8.2 via comunicação Myrinet 10Gbps para compilador da Intel

\end{itemize}
\end{itemize}



% An example of a floating figure using the graphicx package.
% Note that \label must occur AFTER (or within) \caption.
% For figures, \caption should occur after the \includegraphics.
% Note that IEEEtran v1.7 and later has special internal code that
% is designed to preserve the operation of \label within \caption
% even when the captionsoff option is in effect. However, because
% of issues like this, it may be the safest practice to put all your
% \label just after \caption rather than within \caption{}.
%
% Reminder: the "draftcls" or "draftclsnofoot", not "draft", class
% option should be used if it is desired that the figures are to be
% displayed while in draft mode.
%
%\begin{figure}[!t]
%\centering
%\includegraphics[width=2.5in]{myfigure}
% where an .eps filename suffix will be assumed under latex, 
% and a .pdf suffix will be assumed for pdflatex; or what has been declared
% via \DeclareGraphicsExtensions.
%\caption{Simulation results for the network.}
%\label{fig_sim}
%\end{figure}

% Note that the IEEE typically puts floats only at the top, even when this
% results in a large percentage of a column being occupied by floats.


% An example of a double column floating figure using two subfigures.
% (The subfig.sty package must be loaded for this to work.)
% The subfigure \label commands are set within each subfloat command,
% and the \label for the overall figure must come after \caption.
% \hfil is used as a separator to get equal spacing.
% Watch out that the combined width of all the subfigures on a 
% line do not exceed the text width or a line break will occur.
%
%\begin{figure*}[!t]
%\centering
%\subfloat[Case I]{\includegraphics[width=2.5in]{box}%
%\label{fig_first_case}}
%\hfil
%\subfloat[Case II]{\includegraphics[width=2.5in]{box}%
%\label{fig_second_case}}
%\caption{Simulation results for the network.}
%\label{fig_sim}
%\end{figure*}
%
% Note that often IEEE papers with subfigures do not employ subfigure
% captions (using the optional argument to \subfloat[]), but instead will
% reference/describe all of them (a), (b), etc., within the main caption.
% Be aware that for subfig.sty to generate the (a), (b), etc., subfigure
% labels, the optional argument to \subfloat must be present. If a
% subcaption is not desired, just leave its contents blank,
% e.g., \subfloat[].


% An example of a floating table. Note that, for IEEE style tables, the
% \caption command should come BEFORE the table and, given that table
% captions serve much like titles, are usually capitalized except for words
% such as a, an, and, as, at, but, by, for, in, nor, of, on, or, the, to
% and up, which are usually not capitalized unless they are the first or
% last word of the caption. Table text will default to \footnotesize as
% the IEEE normally uses this smaller font for tables.
% The \label must come after \caption as always.
%
%\begin{table}[!t]
%% increase table row spacing, adjust to taste
%\renewcommand{\arraystretch}{1.3}
% if using array.sty, it might be a good idea to tweak the value of
% \extrarowheight as needed to properly center the text within the cells
%\caption{An Example of a Table}
%\label{table_example}
%\centering
%% Some packages, such as MDW tools, offer better commands for making tables
%% than the plain LaTeX2e tabular which is used here.
%\begin{tabular}{|c||c|}
%\hline
%One & Two\\
%\hline
%Three & Four\\
%\hline
%\end{tabular}
%\end{table}


% Note that the IEEE does not put floats in the very first column
% - or typically anywhere on the first page for that matter. Also,
% in-text middle ("here") positioning is typically not used, but it
% is allowed and encouraged for Computer Society conferences (but
% not Computer Society journals). Most IEEE journals/conferences use
% top floats exclusively. 
% Note that, LaTeX2e, unlike IEEE journals/conferences, places
% footnotes above bottom floats. This can be corrected via the
% \fnbelowfloat command of the stfloats package.




\section{Conclusion}
The conclusion goes here.




% conference papers do not normally have an appendix



% use section* for acknowledgment
\ifCLASSOPTIONcompsoc
  % The Computer Society usually uses the plural form
  \section*{Acknowledgments}
\else
  % regular IEEE prefers the singular form
  \section*{Acknowledgment}
\fi


The authors would like to thank...





% trigger a \newpage just before the given reference
% number - used to balance the columns on the last page
% adjust value as needed - may need to be readjusted if
% the document is modified later
%\IEEEtriggeratref{8}
% The "triggered" command can be changed if desired:
%\IEEEtriggercmd{\enlargethispage{-5in}}

% references section

% can use a bibliography generated by BibTeX as a .bbl file
% BibTeX documentation can be easily obtained at:
% http://mirror.ctan.org/biblio/bibtex/contrib/doc/
% The IEEEtran BibTeX style support page is at:
% http://www.michaelshell.org/tex/ieeetran/bibtex/
%\bibliographystyle{IEEEtran}
% argument is your BibTeX string definitions and bibliography database(s)
%\bibliography{IEEEabrv,../bib/paper}
%
% <OR> manually copy in the resultant .bbl file
% set second argument of \begin to the number of references
% (used to reserve space for the reference number labels box)
\begin{thebibliography}{1}

\bibitem{nas}
H.~Kopka and P.~W. Daly, \emph{A Guide to \LaTeX}, 3rd~ed.\hskip 1em plus
  0.5em minus 0.4em\relax Harlow, England: Addison-Wesley, 1999.

\end{thebibliography}



% that's all folks
\end{document}


